\documentclass{article}
\usepackage[utf8]{inputenc}
\usepackage{amsmath}
\usepackage{algpseudocode}
\usepackage[margin=0.9in]{geometry}
\usepackage{paralist}
\usepackage{blindtext}
\usepackage{hyperref}

\hypersetup{
    colorlinks=true,
    linkcolor=blue,
    filecolor=magenta,      
    urlcolor=black
    }

\title{ADA Theory Assignment - 3}
\author{
    \\\vspace{0em} Aakarsh Jain \\\vspace{-0.5em}
    \footnotesize{Roll Number - 2021507}\\\vspace{-0.5em}
    \footnotesize{IIIT - Delhi}\\\vspace{-0.5em}
    \footnotesize{\href{mailto:aakarsh21507@iiitd.ac.in}{\texttt{aakarsh21507@iiitd.ac.in}}}
  \and
    \\\vspace{0em} Siddhant Rai Viksit \\\vspace{-0.5em}
    \footnotesize{Roll Number - 2021565}\\\vspace{-0.5em}
    \footnotesize{IIIT - Delhi}\\\vspace{-0.5em}
    \footnotesize{\href{mailto:siddhant21565@iiitd.ac.in}{\texttt{siddhant21565@iiitd.ac.in}}} 
    \vspace{1em}
}

\date{April 25, 2023}

\begin{document}

\maketitle

\section{Question 1}

\subsection{Problem}
The towns and villages of the Island of Sunland are connected by an extensive rail
network. Doweltown is the capital of Sunland. Due to a deadly contagious disease, recently, few
casualties have been reported in the village of Tinkmoth. To prevent the disease from spreading
to Doweltown, the Ministry of Railway of the Sunland wants to completely cut down the rail
communication between Tinkmoth and Doweltown. For this, they wanted to put traffic blocks
between pairs of rail stations that are directly connected by railway track. It means if there are two
stations $x$ and $y$ that are directly connected by railway line, then there is no station in between $x$
and $y$ in that particular line. If a traffic block is put in the track directly connecting $x$ and $y$, then
no train can move from $x$ to $y$. To minimize expense (and public notice), the authority wants to
put as few traffic blocks as as possible. Note that traffic blocks cannot be put in a station, it has to
be put in a rail-track that directly connects two stations.
\newline
Formulate the above as a flow-network problem and design a polynomial-time algorithm to solve
it. Give a precise justification of the running time of your algorithm.

\subsection{Formulating to Network Flow problem}

Let us formulate the railway network as a graph $G$, with set of vertices as $V$ and edges as $E$. For each city in the railway network, we will create a corresponding vertex in V. Thus $|V| = $ nuumber of cities in the rail network. Now, for the rail lines present in the rail network between cities $x$ and $y$, we will add 2 directed edges in $G$, from $x$ to $y$, and from $y$ to $x$. As all the rail lines are similar in nature, we will arbitrarily assign the capacity $1$ to every edge in $E$.

In the above graph $G$, the vertex corresponding to city of Tinkmoth is the source vertex, and the vertex corresponding to city of Doweltown is the sink vertex. Thus, the graph $G$ is a network flow graph with all the required properties of such a graph.

In the above formulation we assume that every city has exactly $1$ station and there is exactly $1$ rail line between any $2$ cities in the rail network.

\subsection{Reduction to Maximum Flow problem}

We now need to find a set of rail lines, $A$, such that when these rail lines are blocked, towns of Tinkmoth and Doweltown are no longer connected. In our network flow model, the blocking of rail lines between cities $x$ and $y$ can be simulated by removing the edge from $x$ to $y$ and the edge $y$ to $x$. 

Thus, the set $A$ of rail lines can be mapped to another set $B$ of edges in the graph $G$, and removing edges in the $B$ should add another connected component to $G$, such that vertices corresponding to Tinkmoth and Doweltown are in different connected components. Thus, the set $B$ is nothing but a cut in the network graph $G$.

Now, we need to find the size of smallest such set $A$, and thus consequently, also minimize the size of set $B$, hence, we need to find the min cut of the graph $G$.

Let's say that $G$ now has the maximal flow possible. Now for a cut, define the component containing source as $X$ and the one containing sink as $Y$. Also, assume that set of edges $B$ is a min cut of the graph, as required by the question. Since every edge has capacity $1$, using Max-flow Min-cut theorem and Integral flow theorem, every edge in $B$ from a vertex in $X$ to a vertex in $Y$, must have flow of $1$. Also every edge in $B$, directed from a vertex in $Y$ to $X$ must have flow $0$. Thus, the capacity of cut $B$ is equal to number of edges in $B$ from a vertex in $X$ to a vertex in $Y$. 

Now, as $B$ is a min-cut for $G$, removing these vertices will disconnect cities of Tinkmoth and Doweltown, and $B$ is the minimal set possible. 

Using Max-flow Min-cut theorem, the max flow of $G$ is equal to capacity of min-cut $B$, which is in turn equal to number of edges in $B$ from a vertex in $X$ to a vertex in $Y$.

Thus, the max flow of graph $G$ will be equal to the count of minimum edges, removing whom there will be no path from source to sink. 

Thus, if we block the rail lines corresponding to these edges, we will cut off communications between towns of Tinkmoth and Doweltown. Hence, the answer is simply the max-flow of graph $G$.

We can now simply find the max-flow of the graph $G$ using Ford-Fulkerson Algorithm, and return that as our final answer.

\subsection{Time Complexity}

The time complexity of finding Maximum Flow using the Ford-Fulkerson Algorithm is $O(|E|F)$, where $F$ is the maximum flow of the network.

Since, maximum flow possible is at most the number of cities, i.e., the number of vertices, The final time complexity is $O(|E||V|)$

Since, all the edges in the graph have unit capacity, we can also use Dinic's Algorithm to find max-flow (which is also based on Ford-Fulkerson Algorithm) to find the maximum flow in runtime complexity of $O(|E||V|^{\frac{2}{3}})$, which is slightly faster.

\subsection{References}

\begin{itemize}
    \item \href{https://www.inf.ufpr.br/pos/techreport/RT_DINF003_2004.pdf}{Computing the Minimum Cut and Maximum
Flow of Undirected Graphs}
    \item 
    \href{https://cp-algorithms.com/graph/dinic.html#unit-capacities-networks}{Runtime Complexity of Dinic's Algorithm on unit capacity networks}
\end{itemize}

\end{document}