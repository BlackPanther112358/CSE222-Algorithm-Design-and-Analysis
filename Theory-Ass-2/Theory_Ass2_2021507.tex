\documentclass{article}
\usepackage[utf8]{inputenc}
\usepackage{amsmath}
\usepackage{algpseudocode}
\usepackage[margin=0.9in]{geometry}
\usepackage{paralist}
\usepackage{blindtext}
\usepackage{hyperref}

\hypersetup{
    colorlinks=true,
    linkcolor=blue,
    filecolor=magenta,      
    urlcolor=black
    }

\title{ADA Theory Assignment - 2}
\author{
    \\\vspace{0em} Aakarsh Jain \\\vspace{-0.5em}
    \footnotesize{Roll Number - 2021507}\\\vspace{-0.5em}
    \footnotesize{IIIT - Delhi}\\\vspace{-0.5em}
    \footnotesize{\href{mailto:aakarsh21507@iiitd.ac.in}{\texttt{aakarsh21507@iiitd.ac.in}}}
  \and
    \\\vspace{0em} Siddhant Rai Viksit \\\vspace{-0.5em}
    \footnotesize{Roll Number - 2021565}\\\vspace{-0.5em}
    \footnotesize{IIIT - Delhi}\\\vspace{-0.5em}
    \footnotesize{\href{mailto:siddhant21565@iiitd.ac.in}{\texttt{siddhant21565@iiitd.ac.in}}} 
    \vspace{1em}
}

\date{April 08, 2023}

\begin{document}

\maketitle

\section{Question 1}

\subsection{Problem}

The police department in the city of Computopia has made all the streets one-way.
But the mayor of the city still claims that it is possible to legally drive from one intersection to any
other intersection.

\begin{itemize}
    \item Formulate this problem as a graph theoretic problem and explain why it can be solved in linear
time.
    \item Suppose it was found that the mayor’s claim was wrong. She has now made a weaker claim: “if
you start driving from town-hall, navigating one-way streets, then no matter where you reach,
there is always a way to drive legally back to the town-hall. Formulate this weaker property as
a graph theoretic problem and explain how it can be solved in linear time.
\end{itemize}

\subsection{Formulating to Graph problem}

It is assumed that we are provided the input of intersections, and streets joining these intersections. Let's assume that there is $n$ instersections and $m$ edges.

We can then start by labelling the town-hall as $0$, and label the remaining $n - 1$ intersections from $1$ to $n - 1$. We can also similarly label the streets from $0$ to $m - 1$. Now, we can think of each intersection as a vertex in the graph and the one-way streets as edges, directed in direction of the street. We will thus create set $V$ for all the vertices and set $E$ to contain all the edges. Thus, we have obtained a directed graph $G$ using $E$ and $V$.

We will now create an adjacency list $adj$. If there is a street from intersection $u$ to $v$, then we will add $v$ to $adj(u)$. Similarly, we will also create an reversed adjacency list $rev\_adj$ for part (b), i.e., If there is a street from intersection $u$ to $v$, we will add $u$ to $rev\_adj(v)$

\subsection{Approach}

\subsubsection{Part a}

We will use the \underline{Kosaraju's algorithm} to distribute the graph $G$ into its Strongly Connected Components (SCC). Then we will check if all the vertices in set $V$ belong to the same SCC. If yes, then Mayor's claim is true, otherwise it is false.

\subsubsection{Part b}

We will first mark all the vertices reachable from the source using DFS as part of set $A$. We will then reverse all the edges and mark all vertices reachable now as the set $B$. If $A \subset B$, the claim is proved, otherwise it is false.

\subsection{Kosaraju's Algorithm to find SCC}

We can use Kosaraju's Algorithm to find the SCC for each vertex of the graph as discussed in Lecture-12 of the course.

The pseudo-code for the same can be found \href{https://en.wikipedia.org/wiki/Kosaraju%27s_algorithm#The_algorithm}{here}, or a detailed implementation in C++ can be found \href{https://cp-algorithms.com/graph/strongly-connected-components.html}{here} along with basic proof of correctness.

Here we will assume that the function $kosaraju$ implements the above algorithm, and takes an adjacency list as input and returns a vector $scc$ with $scc[u]$ denoting the SCC of vertex $u$.

\subsection{Proof of Correctness}

\subsubsection{Part a}

According to mayor's claim, for any 2 intersection, there must be a path from one to another. In the graph, this translates such that for any $2$ vertices $u$ and $v$, there must be a path from $u$ to $v$ and vice versa. Thus, by definition, $u$ and $v$ must belong to the same SCC. Since this is true for any $2$ vertices, the entire set $V$ must belong to the same SCC, i.e., the graph $G$ has a single SCC containing all the vertices.

To verify the same, we will use the \underline{Kosaraju's algorithm} as described above to find the SCC's and check if all the vertices belong to the same SCC.

\subsubsection{Part b}

Let us denote $P(u \rightarrow v,G)$ to mean there exists a path from $u$ to $v$ in the graph $G$. Let us denote $R(G)$ of a graph $G$ as the graph with the same vertices as $G$ but with all edges reversed. Now, 

\begin{equation}
P(v\rightarrow u,G) \rightarrow P(u \rightarrow v,R(G))
\end{equation}
The mayor's claim is 
\newline
$P(u \rightarrow v,G) \rightarrow P(v \rightarrow u,G) \forall v$. 
\newline
Using (1)
\newline
$P(u \rightarrow v,G) \rightarrow P(u \rightarrow v,R(G)) \forall v$
\newline
We can construct the set $v | P(u \rightarrow v,G)$ and $v | P(u \rightarrow v,R(G))$ using DFS and compare them.

\subsection{Pseudo-Code}
\subsubsection{part a}

\begin{algorithmic}[1]

\Procedure{checkClaim}{$adj$}
\State $n \gets \Call{length}{adj}$
\State $scc \gets \Call{kosaraju}{adj}$
\For{$i$ from $0$ to $n - 1$}
\If{$scc[i] \neq scc[i + 1]$}
\State \Return{$False$}
\EndIf
\EndFor
\State \Return{$True$}
\EndProcedure

\end{algorithmic}

\subsubsection{part b}
\begin{algorithmic}
    \Procedure{DFS}{u,visited, adj}
    \State $visited[u] \gets 1$
    \For{$v$ in $adj[u]$}
    \If{$visited[v] \neq 1$}
    \Call{DFS}{$v$}
    \EndIf
    \EndFor
    \EndProcedure

    \Procedure{Solve}{adj,revadj,v,townhall}
    \State $A \gets [0,..,0]$
    \State $B \gets [0,...,0]$
    \State \Call{DFS}{townhall,A,adj}
    \State \Call{DFS}{townhall,B,revadj}
    \For{$u$ in $[1,..,v]$}
    \If{$A[u]=1$ and $B[u]=0$}
    \Return{False}
    \EndIf
    \EndFor
    \newline
    \Return{True}
    \EndProcedure
    
\end{algorithmic}


\subsection{Time Complexity}

\subsubsection{Part a}

The time complexity for \underline{Kosaraju's algorithm} is $O(|V| + |E|)$, and then we check if all the vertices of the graph belong to the same SCC in $O(V)$, since we require $O(1)$ time to check the SCC any vertex belongs to.

Thus, the total Time Complexity is $O(|V| + |E|)$.

\subsubsection{Part b}
The time complexity for the two DFS calls is $O(|V| + |E|)$, and comparing $A$ and $B$ is $O(|V|)$. Preprocessing $adj$ and $revadj$ is $O(|E|)$. Thus our final time complexity will be $O(|V| + |E|)$

\section{Question 2}

\subsection{Problem}

Given an edge-weighted connected undirected graph $G = (V, E)$ with $n + 20$ edges. Design an algorithm that runs in $O(n)$-time and outputs an edge with smallest weight contained in a cycle of G.

\subsection{Formulating to Graph problem}

We are already provided a graph with $n$ vertices and $n + 20$ edges. We can denote the set of vertices of graph as $V$ and set of edges as $E$. We will thus store the graph using an adjacency list $adj$, where $adj(u)$ is the set of all vertices $v$ such that there exists an edge between $u$ and $v$. We will also use the function $getWeights(u, v)$, which returns the list of weights of all the edges between $u$ and $v$.

\subsection{Approach}

Since the provided graph is an undirected graph, if an edge does not belong to a cycle, then it is a bridge (removing the edge from the graph will cause the number of connected components to increase by 1), and vice versa. 

Thus, we can use a simple DFS - Traversal using the algorithm to find bridges described below to find the edges which belong to a cycle, and then find the least weighted edge among all such edges.

NOTE: The above algorithm assumes that there are no self loops in the graph, i.e, there is no edge which has same endpoints.

\subsection{Finding Bridges}

The algorithm for finding bridges is very similar to that of finding \underline{cut-vertices} already described in tutorials. We will use a DFS traversal to visit every node in the graph. For the purpose of this algorithm, we can start DFS from any node. We will then compute the following:

We will store the start time of DFS Traversal for every node in the array $tin$. We will also initialize $tin$ as $-1$ to indicate that the vertex has not been visited yet in the DFS Traversal.

We can now define $low$ for our DFS tree as:

\begin{equation}
low(u) = min
\left\{
    \begin{array}{lr}
        tin(u)\\
        tin(p) & \forall p \text{ such that} (u, p) \text{ is a back edge}\\
        low(v) & \forall v \text{ such that} (u, v) \text{ is a forward edge}
    \end{array}
\right\}
\end{equation}

Thus, for an edge $(u, v)$ in the DFS tree, it is a bridge iff $low(v) > tin(u)$ 

NOTE that a DFS tree is a directed tree, where edges indicate the direction of DFS traversal.

\subsection{Proof of Correctness}

We need to prove that in a weighted undirected graph $G$, if an edge $a$ is not a bridge, then $a$ must belong to a cycle in graph $G$.

We will prove the above using contradiction, Let's assume that there exists an edge $a$ in graph $G$, which is not a bridge and does not belong to a cycle in graph $G$. Since $a$ is a bridge, there must exist a path $P$ in graph $G$ which connects the vertices $u$ and $v$ and does not contain the edge $a$. But, if we create a path $P'$ by adding edge $a$ to path $P$, then $P'$ forms a cycle. This is a contradiction to our assumption, and thus by way of contradiction, our claim is correct.

We can also prove that if an edge $a$ belonging to undirected graph $G$ is a bridge, then $a$ cannot belong to a cycle. 

We will again use method of contradiction and assume there exists and edge $b$ which is a bridge and belongs to a cycle $C$. Now, let $u$ and $v$ be 2 vertices belonging to $C$. As $u$ and $v$ belong to a cycle, there must be $2$ disjoint paths from $u$ to $v$. Thus, if we remove the edge $a$ from the graph, $u$ and $v$ still remain connected using the other path. Thus, the edge $a$ cannot be a bridge. 

Hence, our algorithm will correctly identify all the edges which belong to cycle, by checking if they are a bridge are not. The final output of the algorithm is then the least weight of all such edges.

\subsection{Pseudo-Code}

\begin{algorithmic}[1]

\State $timer \gets 0$  \Comment{Initialize global variable to store the timer}
\State $minWeight \gets \infty$ \Comment{Initialize global variable to store the minimum required weight}

\Procedure{DFS}{$u, par$}
\State $tin(u) \gets timer$
\State $low(u) \gets timer$
\State $timer \gets timer + 1$
\For{$v \in adj(u)$}
\If{$v = par$} \Comment{Ignore the parent vertex}
\State continue
\ElsIf{$tin(v) = -1$} \Comment{Forward edge}
\State \Call{DFS}{$v$, $u$} 
\State $low(u) = min(low(u), low(v))$
\If{$low(v) \leq tin(u)$}
\State $minWeight \gets min(minWeight, min(\Call{getWeights}{u, v}))$
\EndIf
\Else \Comment{Back edge}
\State $low(u) = min(low(u), tin(v))$
\State $minWeight \gets min(minWeight, min(\Call{getWeights}{u, v}))$
\EndIf
\EndFor
\EndProcedure

\Procedure{getMinWeight}{}
\For{$u \in V$}
\If{$tin(u) = -1$}
\State $\Call{DFS}{u, -1}$
\EndIf
\EndFor
\If{$minWeight = \infty$} \Comment{If there is no cycle in the graph, return $-1$}
\State $minWeight \gets -1$
\EndIf
\State \Return{$minWeight$}
\EndProcedure

\end{algorithmic}


\subsection{Time Complexity}

Since we use a single DFS traversal of the graph, the Time Complexity of the solution is $O(|V| + |E|)$, where $|V|$ and $|E|$ are the number of vertices and edges respectively.

Since we know that $|V| = n$ and $|E| = n + 20$, the Time Complexity of the algorithm can be simplified to $O(n)$.

\subsection{References}

\begin{itemize}
    \item \href{https://cp-algorithms.com/graph/bridge-searching.html}{Algorithm to find bridges}
\end{itemize}

\section{Question - 3}

\subsection{Problem}

Suppose that G be a directed acyclic graph with following features.

\begin{itemize}

\item G has a s  ingle source $s$ and several sinks $t_1,...,t_k$
.
\item Each edge $(v \rightarrow w)$ (i.e. an edge directed from v to w) has an associated weight $Pr(v \rightarrow w)$
between 0 and 1.

\item For each non-sink vertex $v$, the total weight of all the edges leaving v is
$\sum (v \rightarrow w) \in E Pr(v \rightarrow w) = 1$

\item The weights $Pr(v \rightarrow w)$ define a random walk in $G$ from the source $s$ to some sink $t_i$; after
reaching any non-sink vertex $v$, the walk follows the edge $v \rightarrow w$ with probability $Pr(v \rightarrow w)$.

\item All the probabilities are mutually independent.

\end{itemize}

Describe and analyze an algorithm to compute the
probability that this random walk reaches sink $t_i$

for every $i \in \{1,...,k\}$. You can assume that an arithmetic operation takes $O(1)$ time.

\subsection{Pre-Processing}

We will represent the edges $E$ as an array of arrays, with $E_v$ being $I(v)$ ($u \in E_v$ if $(u\rightarrow v)$ is an edge in the graph). This is equivalent to $rev_adj$. We shall also initialize an array $memo$ of size $|V|$ to store all subproblems, and an answer array of size $|t|$ to store the final answers.

\subsection{Approach}

We shall use dynamic programmming, with the state $P(v)$ meaning the probability of a walk reaching vertex $v$, and the transitions given by
$P(v) = \sum_{\exists i \rightarrow v}P(i)\cdot Pr(i \rightarrow v)$

\subsection{Proof of Correctness}
For any vertex $v$, let us denote $I(v)$ as the set of all vertices $x | (x \rightarrow v) \in E$. 
To reach the vertex $v$ on a random walk, the path has to first reach a vertex $i \in I(v)$, then take the edge $(i \rightarrow v)$ to finally get to $v$. Therefore, the probability of reaching any vertex $v$ $P(v)$ is
\newline
$P(v) = \sum_{i \in I(v)}P(i)\cdot Pr(i \rightarrow v)$ 
\newline
We can thus compute $P(t_i) \forall i \in {1,...,k}$ using dynamic programming 

\subsection{Conditions} 
\begin{itemize}
\item The sum of probabilities of outgoing edges from every node is 1, therefore, we can form a complete probability tree for every vertex.
\item The source does not have any incoming edges
\item Since this is a DAG, there aren't any circular dependencies in our recurrence relation
\end{itemize}

\subsection{Subproblem}
Our subproblem $P(v)$ is the probability of reaching the vertex $v$ from $s$.

Therefore the subproblems for the final answers are $P(t_i) \forall i \in {1,...,k}$

\subsection{Recurrence Relation}


$P(v) = \sum_{i \in I(v)}P(i)\cdot Pr(i \rightarrow v)$

with our base case being $P(s) = 1$, since all walks start from there. For any individual path involving the edge $i\in I(v)$, the probability is $P(i)\cdot Pr(i \rightarrow v)$. We can just sum this over all $I(v)$ to get our final $P(v)$.

\subsection{Graph Traversal}

Since we solve every subproblem once, thus the corresponding vertex is picked and its immediate neighbours are visited next, this is equivalent to a DFS traversal of the graph.

\subsection{Pseudo-Code}

\begin{algorithmic}[1]
\State $memo \gets [-1,...,-1]$ 
\Procedure{P}{v,E}
\If{$memo[v] = -1$}
\State $s \gets 0$
\For{$i$ in $E[v]$}
\State $s \gets s + \Call{P}{i,E}\cdot Pr(i,v)$
\EndFor
\State $memo[v] \gets s$
\EndIf

\State \Return{$memo[v]$}
\EndProcedure

\Procedure{solve}{E,s,t}
\State $memo[s] \gets 1$
\State $answer \gets [-1,..,-1]$
\For{$t_i$ in $t$}
\State $answer[i] \gets P(t_i,E)$
\EndFor

\State \Return{$answer$}
\EndProcedure
\end{algorithmic}

\subsection{Time Complexity}

Since there are $V$ subproblems and to solve them we need to iterate through all edges once, the final time complexity is $O(V+E)$

\end{document}
